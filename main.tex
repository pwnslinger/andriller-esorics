% This is samplepaper.tex, a sample chapter demonstrating the
% LLNCS macro package for Springer Computer Science proceedings;
% Version 2.20 of 2017/10/04
%
\documentclass[runningheads]{llncs}
%
\usepackage{graphicx}
% Used for displaying a sample figure. If possible, figure files should
% be included in EPS format.
%
% If you use the hyperref package, please uncomment the following line
% to display URLs in blue roman font according to Springer's eBook style:
% \renewcommand\UrlFont{\color{blue}\rmfamily}

\begin{document}
%
\title{Andriller: A scalable Coverage-guided fuzzing infrastructure for Android system services \thanks{Supported by organization x.}}
%
%\titlerunning{Abbreviated paper title}
% If the paper title is too long for the running head, you can set
% an abbreviated paper title here
%
\author{Mohsen Ahmadi\inst{1} \and
Antonio Bianchi\inst{2}} %\and
%Third Author\inst{3}\orcidID{2222--3333-4444-5555}}
%
\authorrunning{F. Author et al.}
% First names are abbreviated in the running head.
% If there are more than two authors, 'et al.' is used.
%
\institute{Unievrsity of California Los Angeles (UCLA)\\
\email{pwnslinger@ucla.edu} \and
Purdue University\\
\email{antoniob@purdue.edu}}
%
\maketitle              % typeset the header of the contribution
%
\begin{abstract}
The abstract should briefly summarize the contents of the paper in
150--250 words.

\keywords{First keyword  \and Second keyword \and Another keyword.}
\end{abstract}
%
%
%
\section{Introduction}
\section{Motivation}
\subsection{Challenges}
\section{Related Work}
\section{Andriller Architecture}
\subsection{Overview}
\subsection{ARTist}
We build our code instrumentation as modules for the ARTist tool \cite{artist}. We implement two modules for providing feedback from within the systemserver: One for feeding back coverage information to AFL and Troop, and another one for detecting crashes. Further, we patched ARTist such that it can replace the system compiler, which automatically applies our modules to the systemserver upon each fresh emulator start.


Coverage. To provide precise feedback, AFLFuzz’s coverage module injects tracking code at basic blocks granularity. At compiletime, deterministic basic block IDs are generated and embedded into the code so that the injected code can make use of them at runtime. The collected basic block IDs are stored to compute the edge coverage and written to AFL’s shared memory buffer to indicate the path of execution triggered by the current input. In order to reduce the amount of instrumented methods, and therefore improve the precision of our coverage metric and reduce the runtime overhead, we conduct a light-weight static reachability analysis as a pre-processing step. We leverage class-hierarchy information to infer the set of reachable methods per API. We filter invocations of methods that do not reside in Android packages, e.g., classes within the JDK. Further, we follow the approach by axplorer \cite{backes2016demystifying} to stop at IPC boundaries, i.e., we do not descent into public API methods that are fuzzed separately. Not instrumenting the entire systemserver also prevents false positives, e.g., when background tasks are executed by the system and new execution paths are triggered independently from the fuzzing input. The resulting set of reachable methods is subsequently used as a whitelist for the instrumentation, i.e., only reachable methods are modified to return coverage feedback. This implies that we have to start a fresh emulator, because each new API now comes with a pre-computed list of reachable methods that results in a different instrumentation.

\subsection{Coresync}
\subsection{AFLFuzz}
\subsection{Forkserver}
\subsection{Simcheck}
\subsection{Summary}
\section{Evaluation}
\subsection{Setup}
\subsection{Results}
\subsection{Coverage}
\subsection{Performance}
\section{Discussion}
\subsection{Case Study I}
\subsection{Case Study II}
\subsection{Case Study III}
\section{Future Work}
\section{Conclusion}
Please note that the first paragraph of a section or subsection is
not indented. The first paragraph that follows a table, figure,
equation etc. does not need an indent, either.

Subsequent paragraphs, however, are indented.

\subsubsection{Sample Heading (Third Level)} Only two levels of
headings should be numbered. Lower level headings remain unnumbered;
they are formatted as run-in headings.

\paragraph{Sample Heading (Fourth Level)}
The contribution should contain no more than four levels of
headings. Table~\ref{tab1} gives a summary of all heading levels.

\begin{table}
\caption{Table captions should be placed above the
tables.}\label{tab1}
\begin{tabular}{|l|l|l|}
\hline
Heading level &  Example & Font size and style\\
\hline
Title (centered) &  {\Large\bfseries Lecture Notes} & 14 point, bold\\
1st-level heading &  {\large\bfseries 1 Introduction} & 12 point, bold\\
2nd-level heading & {\bfseries 2.1 Printing Area} & 10 point, bold\\
3rd-level heading & {\bfseries Run-in Heading in Bold.} Text follows & 10 point, bold\\
4th-level heading & {\itshape Lowest Level Heading.} Text follows & 10 point, italic\\
\hline
\end{tabular}
\end{table}


\noindent Displayed equations are centered and set on a separate
line.
\begin{equation}
x + y = z
\end{equation}
Please try to avoid rasterized images for line-art diagrams and
schemas. Whenever possible, use vector graphics instead (see
Fig.~\ref{fig1}).

\begin{figure}
\includegraphics[width=\textwidth]{fig1.eps}
\caption{A figure caption is always placed below the illustration.
Please note that short captions are centered, while long ones are
justified by the macro package automatically.} \label{fig1}
\end{figure}

\begin{theorem}
This is a sample theorem. The run-in heading is set in bold, while
the following text appears in italics. Definitions, lemmas,
propositions, and corollaries are styled the same way.
\end{theorem}
%
% the environments 'definition', 'lemma', 'proposition', 'corollary',
% 'remark', and 'example' are defined in the LLNCS documentclass as well.
%
\begin{proof}
Proofs, examples, and remarks have the initial word in italics,
while the following text appears in normal font.
\end{proof}
For citations of references, we prefer the use of square brackets
and consecutive numbers. Citations using labels or the author/year
convention are also acceptable. The following bibliography provides
a sample reference list with entries for journal
articles~\cite{ref_article1}, an LNCS chapter~\cite{ref_lncs1}, a
book~\cite{ref_book1}, proceedings without editors~\cite{ref_proc1},
and a homepage~\cite{ref_url1}. Multiple citations are grouped
\cite{ref_article1,ref_lncs1,ref_book1},
\cite{ref_article1,ref_book1,ref_proc1,ref_url1}.
%
% ---- Bibliography ----
%
% BibTeX users should specify bibliography style 'splncs04'.
% References will then be sorted and formatted in the correct style.
%
\bibliographystyle{splncs04}
\bibliography{references}
%
\end{document}
